%% Exemplo de utilizacao do estilo de formatacao normas-utf-tex (http://normas-utf-tex.sourceforge.net)
%% Autores: Hugo Vieira Neto (hvieir@utfpr.edu.br)
%%          Diogo Rosa Kuiaski (diogo.kuiaski@gmail.com)
%% Colaboradores:
%%          Cézar M. Vargas Benitez <cesarvargasb@gmail.com>
%%          Marcos Talau <talau@users.sourceforge.net>


\documentclass[openright]{normas-utf-tex} %openright = o capitulo comeca sempre em paginas impares
%\documentclass[oneside]{normas-utf-tex} %oneside = para dissertacoes com numero de paginas menor que 100 (apenas frente da folha) 


\usepackage[alf,abnt-emphasize=bf,bibjustif,recuo=0cm, abnt-etal-cite=2, abnt-etal-list=99]{abntcite} %configuracao correta das referencias bibliograficas.

\usepackage[brazil]{babel} % pacote portugues brasileiro
\usepackage[utf8]{inputenc} % pacote para acentuacao direta
\usepackage{amsmath,amsfonts,amssymb} % pacote matematico
\usepackage{graphicx} % pacote grafico
\usepackage{times} % fonte times
\usepackage{listings} % para código-fonte
\usepackage{algorithmic}

% configuracao do pacote listings
\lstset{
	basicstyle=\footnotesize,
	showstringspaces=false,
	tabsize=4,
	numbers=left,
	numberstyle=\tiny,
	stepnumber=1,
	numbersep=5pt,
	breaklines=true,
	extendedchars=true,
	frame=tb
}

%Podem utilizar GEOMETRY{...} para realizar pequenos ajustes das margens. Onde, left=esquerda, right=direita, top=superior, bottom=inferior. P.ex.:
%\geometry{left=3.0cm,right=1.5cm,top=4cm,bottom=1cm} 

% ---------- Preambulo ----------
\instituicao{Universidade Tecnol\'ogica Federal do Paran\'a} % nome da instituicao
\departamento{Departamento Acadêmico de Informática}
\programa{Curso Superior de Engenharia de Computação} % nome do programa

\documento{Trabalho de Conclusão de Curso} % [Disserta\c{c}\~ao] ou [Tese]
\nivel{Graduação} % [Mestrado] ou [Doutorado]
\titulacao{Engenheiro} % [Mestre] ou [Doutor]

\titulo{\MakeUppercase{Titulo}} % titulo do trabalho em portugues
\title{\MakeUppercase{Title}} % titulo do trabalho em ingles

\autor{Brunno Alberto Wistuba Braga} % autor do trabalho
\autordois{Caio Nogara Andreatta}
\cita{ANDREATTA, Caio N.; BRAGA, Brunno A. W.} % sobrenome (maiusculas), nome do autor do trabalho

\palavraschave{vídeo digital, geração artificial de artefatos, avaliação de qualidade de vídeo, avaliação subjetiva, avaliação objetiva} % palavras-chave do trabalho
\keywords{Keywords} % palavras-chave do trabalho em ingles

\comentario{\UTFPRdocumentodata\ de graduação, apresentado à disciplina de Trabalho de Conclusão de Curso II, do \UTFPRprogramadata\ do \UTFPRdepartamentodata\ da \ABNTinstituicaodata\, como requisito parcial para obten\c{c}\~ao do grau de \UTFPRtitulacaodata.}



%\orientador{Prof. Dra. Keiko Veronica Ono Fonseca} % nome do orientador do trabalho
\orientador[Orientadora:]{Prof\textsuperscript{a}. Dr\textsuperscript{a}. Keiko Veronica Ono Fonseca} % <- no caso de orientadora, usar esta sintaxe
%\coorientador{Nome do Co-orientador} % nome do co-orientador do trabalho, caso exista
%\coorientador[Co-orientadora:]{Nome da Co-orientadora} % <- no caso de co-orientadora, usar esta sintaxe
%\coorientador[Co-orientadores:]{Nome do Co-orientador} % no caso de 2 co-orientadores, usar esta sintaxe
%\coorientadorb{Nome do Co-orientador 2}	% este comando inclui o nome do 2o co-orientador

\local{Curitiba} % cidade
\data{\the\year} % ano automatico


%---------- Inicio do Documento ----------
\begin{document}

\capa % geracao automatica da capa
\folhaderosto % geracao automatica da folha de rosto
%\termodeaprovacao % <- ainda a ser implementado corretamente

% agradecimentos (opcional)
\begin{agradecimentos}
Agradecimentos.
\end{agradecimentos}

%resumo
\begin{resumo}
Este projeto teve como objetivo aperfeiçoar o \emph{software} SASQV de avaliação subjetiva e objetiva da qualidade de vídeos digitais, assim como desenvolver e integrar, em uma plataforma unificada de avaliação de qualidade, ferramentas para a adição controlada de artefatos em vídeos. A manipulação destes artefatos por meio de parametrização, permitem a adição de  efeitos de borramento e blocagem, assim como permitem a simulação de artefatos decorrentes da perda de pacotes em uma transmissão de vídeo por \emph{streaming}.
\end{resumo}

%abstract
\begin{abstract}
This project aimed to improve the SASQV software of subjective and objective video quality assessment, as well as to develop and integrate, into an unified platform for assessing video quality, tools for adding artifacts in a controlled manner. The manipulation of these artifacts by means of parameterization allows the addition of blocking and blurring effects and the simulation of artifacts resulting from packet loss in a transmission of streaming video.
\end{abstract}

% listas (opcionais, mas recomenda-se a partir de 5 elementos)
\listadefiguras % geracao automatica da lista de figuras
\listadetabelas % geracao automatica da lista de tabelas
\listadesiglas % geracao automatica da lista de siglas

% sumario
\sumario % geracao automatica do sumario


%---------- Inicio do Texto ----------
% recomenda-se a escrita de cada capitulo em um arquivo texto separado (exemplo: intro.tex, fund.tex, exper.tex, concl.tex, etc.) e a posterior inclusao dos mesmos no mestre do documento utilizando o comando \input{}, da seguinte forma:

% aparentemente esse sloppy resolve o problema das margens e do texttt
% não tenho certeza sobre os efeitos colaterais dele, no entanto
% fiquemos atentos! :P
\sloppy

\chapter{Introdução}

Contando com os avanços tecnológicos das últimas décadas, o vídeo digital se encontra cada vez mais presente no dia a dia das pessoas ao redor do mundo, seja como educação, entretenimento ou informação.
Além de acessível, também é necessário que o vídeo chegue a essas pessoas com um nível de qualidade que as possibilitem contemplar e interpretar as imagens de forma natural, sem produzir degradação ou desconforto perceptível ao Sistema Visual Humano (\sigla{SVH}{Sistema Visual Humano}), ou seja, sem que existam perdas ou distorções exageradas em seu conteúdo.
No entanto, processos como aquisição, compressão, armazenamento e transmissão, os quais viabilizam e popularizam a difusão de vídeo, são muitas vezes responsáveis por também introduzir artefatos que degradam a qualidade final da imagem \cite{daronco}.

No intuito de melhor avaliar o efeito que tais artefatos podem produzir no SVH foram desenvolvidas métricas objetivas e subjetivas de avaliação de qualidade de vídeo. O projeto SASQV, desenvolvido por \cite{sasqv}, busca implementar ambas as formas de avaliação em um mesmo conjunto de \emph{software} e \emph{hardware} onde é possível gerar artefatos artificiais, coletar avaliações subjetivas e aplicar métricas objetivas sobre uma base de vídeos, também ferramentas para análise e comparação dos dados obtidos. O presente trabalho propõe adaptações ao SASQV, buscando aperfeiçoar a ferramenta de  geração de artefatos no sentido de torná-los mais similares àqueles encontrados nas trasmissões, fornecendo maior grau de liberdade para a manipulação dos mesmos, além de agregar novos tipos, como por exemplo a simulação de um \emph{streaming} de vídeo. Dada a natureza do projeto SASQV, o qual se utiliza de softwares e bibliotecas distribuidos sob licenças de \emph{software} livre, este projeto também se propõe a construir uma nova interface gráfica não baseada em tecnologias proprietárias.

A proposta é motivada pela ampla difusão de vídeo digital, implantadas principalmente na forma de TV digital e de \emph{streaming} via internet, sendo o segundo objeto de grande interesse no mercado. Segundo \cite{sandvinereport}, no continente norte-americano cerca de 37\% do tráfego de internet fixa é dedicado ao \emph{streaming} de vídeo durante o horário onde tradicionalmente ocorre o pico de audiência televisiva, atingindo 41\% no caso da internet móvel dentro do mesmo horário.

\section{Palavras-Chave}
Vídeo digital, Geração de artefatos de vídeo digital, Avaliação subjetiva e objetiva de qualidade de vídeo.
\section{Objetivo Geral}
Adaptar a ferramenta \sigla{SASQV}{Sistema de Avaliação Subjetiva de Qualidade de Vídeo} no sentido de aprimorar a degradação, reprodução, manipulação, interação e portabilidade  presentes na ferramenta original.
\section{Objetivos Específicos}
\begin{itemize}
	\item \textbf{} Aprimorar os algoritmos de geração de artefatos (\emph{blocking} e \emph{blurring} - blocagem e borramento) de vídeo digital, permitindo ao usuário manipular os parâmetros de cada artefato.
	\item \textbf{} Adaptar a ferramenta original para manipular vídeo bruto no formato \emph{.yuv}.
	\item \textbf{} Adicionar à ferramenta um simulador de transmissões de vídeo via rede por meio de \emph{streaming} e seus possíveis artefatos.
	\item \textbf{} Desenvolver uma nova interface gráfica em linguagem Java.
	\item \textbf{} Aprimorar a portabilidade da ferramenta para sistemas operacionais diversos.
\end{itemize}


\section{Fundamentação}

% ------ estrutura
% 
\begin{frame}\frametitle{Introdução}
\end{frame}

\subsection{Vídeo Digital}
    \begin{frame}\frametitle{}
    \end{frame}

\subsection{Codificação}
    \begin{frame}\frametitle{}
    \end{frame}

\subsection{Artefatos}
    \begin{frame}\frametitle{}
    \end{frame}

\subsection{Métricas de Avaliação}
    \begin{frame}\frametitle{}
    \end{frame}
    
    \subsubsection{Métricas Objetivas}
        \begin{frame}\frametitle{}
        \end{frame}
        
    \subsubsection{Métricas Subjetivas}
        \begin{frame}\frametitle{}
        \end{frame}
    
\subsection{Distribuições de Probabilidade}
    \begin{frame}\frametitle{}
    \end{frame}
    
    \subsubsection{Distribuição Uniforme}
        \begin{frame}\frametitle{}
        \end{frame}
        
    \subsubsection{Distribuição Normal}
        \begin{frame}\frametitle{}
        \end{frame}
        
    \subsubsection{Distribuição Triangular}
        \begin{frame}\frametitle{}
        \end{frame}
  


%----------- Terceiro Capítulo: Metodologia --------------

\chapter{Especificação} %(ou especifiação) 10 --- 20 pags

O \emph{software} a ser desenvolvido neste trabalho deverá fornecer uma ferramenta flexível de degradação de vídeo digital e também de avaliação subjetiva e objetiva das mídias.
Neste capítulo serão apresentados os requisitos de tal sistema, assim como as especificações de desenvolvimento, arquitetura e funcionamento.

\section{Análise de Requisitos}

A análise de requisitos é parte fundamental de qualquer projeto, buscando atender da melhor forma as necessidades dos futuros usuários.
Esta sessão descreve os requisitos levantados após o estudo e planejamento da ferramenta.

\subsection{Requisitos Funcionais}

Requisitos funcionais são aqueles determinados pelas funcionalidades exigidas do sistema, ou seja, as ações que se espera que o sistema execute. 
Foram levantados os seguintes requisitos funcionais:

\begin{itemize}
	\item O \emph{software} deverá possuir ferramentas de degradação de vídeos digitais.
	\item O \emph{software} deverá possuir uma ferramenta de simulação de transmissão \emph{streaming}.
	\item O \emph{software} deverá ser capaz de criar e gerenciar sessões de avaliação subjetiva.
	\item O \emph{software} deverá ser capaz de exibir os vídeos existentes na base de dados.
	\item O \emph{software} deverá exibir os resultados das avaliações objetivas e subjetivas de forma gráfica.
\end{itemize}

\subsection{Requisitos Não-Funcionais}

Requisitos não-funcionais são aqueles ditados por restrições ou exigências de qualidade ou de operação, tais como performance, segurança ou tecnologias envolvidas.
Abaixo se encontram os requisitos não funcionais levantados:

% TODO dividir por categorias: padronização, usabilidade, tecnologias envolvidas.

\begin{itemize}
	\item As ferramentas de degradação e métricas objetivas deverão operar sobre vídeos em formato YUV planificado com subamostragem 4:2:0.
	\item A ferramenta de simulação deverá operar sobre vídeos no formato \sigla{H.262}{Padrão de compressão de vídeo digital também conhecido MPEG-2 Part 2} encapsulados em um Transport Stream (\sigla{TS}{Transport Stream}).
	\item O sistema deve fornecer documentação detalhada de auxílio ao uso das ferramentas.
	\item A ferramenta de exibição deverá ser executada em um sistema capaz de prover um \emph{display} com resolução igual ou maior que a dos vídeos a serem exibidos.
	% TODO verificar a versão da JRE necessária para rodar a interface
	\item A interface gráfica do sistema necessita que haja um \emph{Java Runtime Environment} (\sigla{JRE}{Java Runtime Environment}) instalado no sistema operacional.
\end{itemize}

\section{Especificações do Software}

Esta seção tem o propósito de descrever a arquitetura proposta para o SASQV2, assim como abordar as diferentes linguagens e bibliotecas que ajudam a integrar o programa.

\subsection{Arquitetura do Sistema}

Por se tratar da continuação do desenvolvimento do projeto SASQV, este trabalho reutiliza e adapta grande parte dos componentes existentes no trabalho original. 
A Figura \ref{fig:arquitetura} mostra uma comparação entre as visões gerais de ambos os projetos. A diferença mais notável entre estas arquiteturas é a separação dos componentes de serviço e o de ferramentas, além de também recriar o componente de interface gráfica para melhor atender às necessidades do sistema. 
Mais detalhes sobre cada componente da arquitetura são descritos em seções específicas a seguir.

\begin{figure}[!htb]
	\centering
	\includegraphics[width=0.9\textwidth]{./imgs/arquitetura.png}
	\caption{Visão geral das arquiteturas dos sistemas SASQV e SASQV2.}
	\label{fig:arquitetura}
	\fonte{Autoria Própria.}
\end{figure}

\subsubsection{Banco de Dados}

O componente banco de dados foi completamente reaproveitado, seguindo as mesma especificações do SASQV. 
O Sistema Gerenciador de Banco de Dados (\sigla{SGBD}{Sistema Gerenciador de Banco de Dados}) escolhido foi o MySQL, atualmente o SGDB \emph{open source} mais utilizado no mundo \cite{mysqlmarket}.
O MySQL foi lançado e desenvolvido em 1995 pela empresa Sueca MySQL AB, atualmente incorporada pela Oracle Corporation, na forma de um SGDB que fornece um servidor multi-usuários para bancos de dados relacionais \cite{wikipediamysql}. 
Juntamente do MySQL também é utilizado o MySQL Workbench, uma ferramenta distribuida pela Oracle que concilia um cliente MySQL, uma ferramenta de \emph{Database Moddeling} e um administrador de servidor MySQL. 
A versão \emph{open source} de ambas as ferramentas é distribuída sob a licensa GNU General Public License (\sigla{GPL}{General Public License}).

O modelo Entidade-Relacionamento (\sigla{ER}{Entidade-Relacionamento}) é o mesmo utilizado pelo SASQV, como mostrado na Figura \ref{fig:diagramaER}.

\begin{figure}[!htb]
	\centering
	\includegraphics[width=0.9\textwidth]{./imgs/diagramaER.png}
	\caption{Diagrama Entidade-Relacionamento do SASQV2}
	\label{fig:diagramaER}
	\fonte{Autoria Própria}
\end{figure}

\subsubsection{Model}

O componente Model presente no SASQV2 também foi reaproveitado inteiramente do SASQV, cabendo apenas algumas adições. Este componente consiste na implementação de um mapeamento objeto-relacional (\sigla{MOR}{Mapeamento Objeto Relacional}) por meio da biblioteca Hibernate. 
O Hibernate teve seu desenvolvimento iniciado em 2001 por Gavin King, com o objetivo de melhorar as ferramentas de persistência existentes na plataforma Java \cite{hibernateHistory}, e é atualmente distribuído sob a licensa GNU Lesser General Public License (\sigla{LGPL}{Lesser General Public License}) \cite{hibernateAbout}.
A biblioteca provê um mapeamento flexível entre objetos Java e tipos SQL, eliminando a custosa necessidade de processamento manual e repetitivo de listas de resultados obtidas via SQL e JDBC.
Estima-se que até cerca de 30\% do código de uma aplicação possa ser poupado utilizando MOR, além de incrementar a portabilidade do sistema suportando diversas implementações de Bancos de Dados SQL em troca de um pequeno \emph{overhead} de performance.

\subsubsection{Service}

Este elemento também foi reaproveitado, no entanto sofreu diversas adaptações em sua estrutura para comportar novas funcionalidades e continuar suportando as antigas.
O componente presente no SASQV se trata de uma fachada de serviço responsável por interpretar mensagens provenientes da interface e distribuí-las em uma das três formas a seguir:

\begin{enumerate}
	\item Executar uma ação de degradação ou métrica objetiva.
	\item Repassar uma ação para o dispositivo sem-fio e esperar por resposta.
	\item Repassar uma ação para o elemento model e esperar por resposta.
\end{enumerate}

Na arquitetura do SASQV2 a interpretação de mensagens foi descartada, uma vez que foi prosposta a implementação de uma nova interface em linguagem Java, e deu lugar a métodos de finalidades distintas dentro de cada ação enumerada acima. % TODO verificar se o acima pode virar abaixo?!!?
Com isso execução de tarefas de degradação ou avaliação por métrica objetiva, antes executadas localmente, passaram a ser delegadas para um novo componente chamado Ferramentas, descrito em \ref{met:ferramentas}.
Já o conteúdo dos métodos que repassam as ações para o dispositivo sem-fio e para o model permanecem muito semelhantes aos encontrados na interpretação de mensagens originalmente feito no SASQV, sofrendo apenas pequenas modificações para acomodar a nova estrutura.

\subsubsection{Ferramentas}
\label{met:ferramentas}

\subsubsection{Interface Gráfica}
\subsection{Linguagens}
\subsection{Bibliotecas}
\subsection{Diagramas de Caso de Uso}
\section{Considerações}


\section{Desenvolvimento}

% ------ estrutura
\subsection{Ferramentas Utilizadas}
    \begin{frame}\frametitle{}
    % ide, g++, make, gui builder
    \end{frame}
    
\subsection{Ferramentas Desenvolvidas}
    \begin{frame}\frametitle{}
    \end{frame}
    
    \subsubsection{\emph{raffle}}
        \begin{frame}\frametitle{}
        \end{frame}
    
    \subsubsection{\emph{block}}
        \begin{frame}\frametitle{}
        \end{frame}
        
    \subsubsection{\emph{blur}}
        \begin{frame}\frametitle{}
        \end{frame}
        
    \subsubsection{\emph{NetSim}}
        \begin{frame}\frametitle{}
        \end{frame}
        
    \subsubsection{\emph{metric}}
        \begin{frame}\frametitle{}
        \end{frame}

\subsection{Interface Gráfica}
    \subsubsection{Sessão}
        \begin{frame}\frametitle{}
        \end{frame}
        
    \subsubsection{Ferramentas}
        \begin{frame}\frametitle{}
        \end{frame}
        
        \subsubsubsection{Gerador de Artefatos}
            \begin{frame}\frametitle{}
            \end{frame}
            
        \subsubsubsection{Avaliador Objetivo}
            \begin{frame}\frametitle{}
            \end{frame}
            
        \subsubsubsection{Gerador de Aleatoriedade}
            \begin{frame}\frametitle{}
            \end{frame}
            
    \subsubsection{Utilidades}
        \begin{frame}\frametitle{}
        \end{frame}
        \subsubsubsection{RaffleViewer}
            \begin{frame}\frametitle{}
            \end{frame}
            
        \subsubsubsection{MPlayer}
            \begin{frame}\frametitle{}
            \end{frame}


%---------- Quinto Capítulo: Resultados ----------

\chapter{Resultados} %5 --- 10 pags

Com o conjunto de ferramentas desenvolvidos neste projeto aliado a uma base de vídeos originais é possível produzir e avaliar uma nova e diversificada base com vídeos degradados em qualidade e quantitade variável.
Neste Capítulo serão verificados os resultados obtidos a partir do processamento de vídeos efuados por estas ferramentas. Primeiramente uma avaliação subjetiva das ferramentas de degradação utilizando diversas amostras retiradas de um vídeo da base. A seguir é feita a validação dos resultados obtidos pela ferramenta de métricas objetivas, tomando como base os videos obtidos em \cite{xxx}.
Por fim é feita uma análise das notas de avaliação objetivas buscando mensurar o impacto que determinadas degradações podem ter sobre um conjunto de vídeos com características diferentes.

\section{Validação das Ferramentas de Artefatos}

Foram efetuados diversos processos de degradação sobre o vídeo bus, obtido em \cite{tracevideoseq},  empregando diferentes paramêtros a cada processamento e para cada ferramenta, buscando identificar visualmente e os artefatos produzidos e determinar sua semelhanca com os vistos em situações reais.

A primeira ferramenta a ser avaliada é a \emph{block}.
Para a avaliação foram gerados 14 vídeos onde é efetuado o processo de blocagem, com blocos 8x8, de forma integral --- para todos os blocos em todos os quadros. A cada novo vídeo gerado foram eliminadas de forma cumulativa diagonais na matriz resultante da transformação DCT, partindo da diagonal mais inferior, até que no último vídeo restasse somente a componente DC da transformada. 
Na Figura \ref{fig:blockbus} são apresentados recortes de 160 por 160 \emph{pixels} a partir da posição 64x160 retirados do \emph{frame} de número 60 de cada um dos 14 vídeos obtidos, sendo que o primeiro recorte foi obtido do vídeo original.

Observando os recortes com cautela é possível notar degradações extremamente sutis a partir do recorte \emph{07}, verificadas com mais facilidade nos arredores da publicidade na lateral do ônibus.No recorte de número \emph{10} as regiões de alta frequência da imagem denunciam degradações mais acentuadas, e no \emph{11} já não é mais possível identificar o que está escrito na placa de publicidade, embora ainda seja razoavél identificar a cena da imagem.
Nos três recortes seguintes o efeito de blocagem passa a tomar conta da imagem e apenas objetos de grande escala passam a ser identificávies, sendo que no recorte de número \emph{14} até mesmo a identificação da cena fica prejudicada.

\begin{figure}[!htb]
	\centering
	\includegraphics[width=0.8\textwidth]{./imgs/blockbus.png}
	\caption{Sequência de degradações eliminando gradativamente diagonais da DCT.}
	\label{fig:blockbus}
	\fonte{Autoria Própria.}
\end{figure}

Para a avaliação da ferramenta \emph{blur} foram feitas duas sequências de testes, a primeira delas utilizando o filtro de médias e a segunda utilizando o filtro da mediana.
Em cada um dos testes foram utilizadas máscaras matriciais de tamanhos 3x3, 5x5 e 7x7. Nas Figuras \ref{fig:bluraverage} e \ref{fig:blurmedian} são apresentados recortes de 160 por 160 \emph{pixels} a partir da posição 64x160 retirados do \emph{frame} número 60 de cada vídeo.

\begin{figure}[!htb]
	\centering
	\includegraphics[width=0.9\textwidth]{./imgs/bluraverage.png}
	\caption[Sequência de borramentos aplicando filtro da média]{Sequência de borramentos aplicando filtro da média com matrizes (00) 3x3, (01) 5x5, (02) 7x7.}
	\label{fig:bluraverage}
	\fonte{Autoria Própria.}
\end{figure}

\begin{figure}[!htb]
	\centering
	\includegraphics[width=0.9\textwidth]{./imgs/blurmedian.png}
	\caption[Sequência de borramentos aplicando filtro da mediana]{Sequência de borramentos aplicando filtro da mediana com matrizes (00) 3x3, (01) 5x5, (02) 7x7.}
	\label{fig:blurmedian}
	\fonte{Autoria Própria.}
\end{figure}

Na validação visual da ferramenta \emph{raffle} foi gerado um arquivo de sorteio contendo 5 mil elementos.
A distribuição no tempo foi configurada como triangular no intervalo de 0 a 10 com pico no \emph{frame} 5, sendo que a duração de cada artefato foi distribuída uniformemente no intervalo de 1 a 3 \emph{frames}.
Já a distribuição no espaço foi uniforme em toda a largura e altura do \emph{frame}. 
A figura \ref{fig:busraffle} mostra a sequencia de recortes retirados dos 11 \emph{frames} contendo blocos gerados a partir do arquivo raffle obtido com a configuração acima, onde para facilitar a visualização foi eliminada a componente DC da DCT para cada bloco sorteado.

\begin{figure}[!htb]
	\centering
	\includegraphics[width=0.8\textwidth]{./imgs/busraffle.png}
	\caption[Sequência de \emph{frames} degradados]{Sequência de \emph{frames} degradados a partir de um arquivo gerado pela ferramenta \emph{raffle}.}
	\label{fig:busraffle}
	\fonte{Autoria Própria.}
\end{figure}

Para o processo de validação da ferramenta \emph{netsim} foi necessário obter um vídeo encapsulado em um TS, para isso foi utilizada a ferramenta ffmpeg para converter o vídeo bus para a codificação MPEG-2 encapsulado em um TS com o seguinte comando:

{
	\centering
	\texttt{ffmpeg -i \emph{arquivo\_de\_entrada} -s cif -sameq -f mpegts -vcodec mpeg2video \emph{arquivo\_de\_saída}}
}

O TS resultante foi então processado pela ferramenta, configurada para fazer os seguintes descartes:

\begin{center}
	\texttt{10 100
	1000 10
	100 10
	1 10
	100 50}
\end{center}

A figura \ref{fig:netsim} contém amostras retiradas dos \emph{frames} 1, 9 e 40, onde podem ser verificados artefatos como \emph{jerkiness}, \emph{blocking} e \emph{bleeding}.

\begin{figure}[!htb]
	\centering
	\includegraphics[height=0.9\textheight]{./imgs/netsimresult.png}
	\caption{Amostra de artefados obtidos com a ferramenta \emph{netsim}}
	\label{fig:netsim}
	\fonte{Autoria Própria.}
\end{figure}

\section{Validação da Ferramenta de Métricas}

Para validação dos valores de MSE, PSNR e MSSIM calculados pela ferramenta \emph{metric} foi feita a comparação destes com os obtidos pela ferramenta MSU \emph{Video Quality Measurement Tool}.
%TODO base de videos das metricas
Foram utilizados 5 vídeos diferentes, e os respectivos vídeos degradados,  obtidos em \cite{xxx}. Os resultados das comparações se encontram nas tabelas \ref{res:mse}, \ref{res:psnr} e \ref{res:mssim}.

\begin{table}[!htb]
	\centering
	\caption{Comparação dos resultados de PSNR.}
	\label{res:psnr}
	\begin{tabular}{lccc}
		\hline
		Video	 & MSU	 & SASQV2	 & Erro (\%) \\ \hline
		aircraft	 & 23.49501 & 23.49501 & 0.00000	 \\ 
		liberty	 & 29.21472 & 29.21472 & 0.00000	 \\ 
		ship	 & 25.56667 & 25.56667 & 0.00000	 \\ 
		stockholm & 18.90942 & 18.90942 & 0.00000	 \\ 
		whale	 & 20.59830 & 20.59830 & 0.00000	 \\
	\hline
	\end{tabular}
	\fonte{Autoria pŕopria}
\end{table}

\begin{table}[!htb]
	\centering
	\caption{Comparação dos resultados de MSE.}
	\label{res:mse}
	\begin{tabular}{lccc}
	\hline
	Video	 & MSU	 & SASQV2	 & Erro (\%) \\ \hline
	aircraft	 & 290.78970 & 290.78970 & 0.00000	 \\ 
	liberty	 & 77.91273	 & 77.91273	 & 0.00000	 \\ 
	ship	 & 180.47363 & 180.47362 & -0.00001 \\ 
	stockholm & 835.86987 & 835.86987 & 0.00000	 \\ 
	whale	 & 566.53094 & 566.56611 & 0.00621	 \\
	\hline
	\end{tabular}
	\fonte{Autoria própria}
\end{table}

\begin{table}[!htb]
	\centering
	\caption{Comparação dos resultados de MSSIM.}
	\label{res:mssim}
	\begin{tabular}{lccc}
	\hline
	Video	 & MSU	 & SASQV2	 & Erro (\%) \\ \hline
	aircraft	 & 0.79457 & 0.80723 & 1.59306	 \\
	liberty	 & 0.85415 & 0.85627 & 0.24843	 \\
	ship	 & 0.70462 & 0.70628 & 0.23530	 \\ 
	stockholm & 0.42486 & 0.43327 & 1.97948	 \\ 
	whale	 & 0.81918 & 0.82240 & 0.39320	 \\
	\hline
	\end{tabular}
	\fonte{Autoria própria.}
\end{table}

Observando os resultados para MSE e PSNR pode-se dizer que os resultados são fiéis e que a ferramenta se comporta como o esperado.
A respeito do MSSIM é difícil dizer ao certo o motivo da pequena diferença encontrada nos resultados, especialmente pela variedade de parâmetros a serem levados em conta nesta métrica. 
Levando em conta a documentação do MSU sabe-se que este utiliza pesos uniformes para toda a janela de cálculo, mas não se encontrou o tamanho adotado ou o método de escolha das janelas possíveis.
O teste utilizando a ferramenta presente no SASQV2 utiliza janelas 8x8 adotando janelas não sobrepostas.

\section{Análise de Impacto Sobre Métricas Objetivas}



Para este teste foi utilizada a combinação de três distribuições de artefatos geradas pela ferramenta raffle, que foram concatenadas para criar três rajadas de ruídos em um conjunto de vídeos. As configurações adotadas foram as seguintes:

\begin{enumerate}
	\item Distribuição 1
	\begin{itemize}
		\item Duração: uniforme de 1 a 5
		\item Frames: normal com \(\mu = 25\) e \(\sigma = 10\)
		\item largura: uniforme de 1 a 10
		\item altura: uniforme de 20 a 30
	\end{itemize}
	\item Distribuição 2
	\begin{itemize}
		\item Duração: constante em 3
		\item Frames: uniforme de 70 a 100
		\item largura: uniforme de 10 a 40
		\item altura: uniforme de 10 a 20
	\end{itemize}
	\item Distribuição 3
	\begin{itemize}
		\item Duração: uniforme de 3 a 5
		\item Frames: normal com \(\mu = 220\) e \(\sigma = 40\)
		\item largura: uniforme de 1 a 43
		\item altura: normal com \(\mu = 28\) e \(\sigma = 5\)
	\end{itemize}
\end{enumerate}

O número de artefados em cada \emph{frame} obtido a partir desta configuração é apresentado na figura \ref{fig:histogram}.

Esta configuração foi então usada na aplicação de blocagem em três vídeos diferentes: Akiyo, CoastGuard e Foreman, todos possuindo 300 \emph{frames} de duração e obtidos em \cite{xiph}. Estes vídeos foram escolhidos pelas suas características de textura e movimento, possibilitando uma análise variada.

Nas figuras \ref{fig:mse} e \ref{fig:mssim} pode-se acompanhar a evolução das métricas MSE e MSSIM quadro a quadro para os três vídeos degradados. Observando o comportamento das métricas nota-se que, apesar da natureza e da distribuição dos artefatos utilizados ser exatamente a mesma para todos os vídeos, os valores obtidos possuem picos diferentes, embora o comportamento seja semelhante.

Outro fato notável é a análise individual das métricas, que dependendo do conjunto de \emph{frames} análisado pode levar a conclusões diferentes a respeito do nível de degradação encontrado em cada vídeo.

\begin{figure}[!htb]
	\centering
	\includegraphics[width=0.9\textwidth]{./imgs/histogram.png}
	\caption{Histograma dos artefatos produzidos}
	\label{fig:histogram}
	\fonte{Autoria Própria.}
\end{figure}

\begin{figure}[!htb]
	\centering
	\includegraphics[width=0.9\textwidth]{./imgs/mse.png}
	\caption{MSE quadro a quadro}
	\label{fig:mse}
	\fonte{Autoria Própria.}
\end{figure}

\begin{figure}[!htb]
	\centering
	\includegraphics[width=0.9\textwidth]{./imgs/mssim.png}
	\caption{MSSIM quadro a quadro}
	\label{fig:mssim}
	\fonte{Autoria Própria.}
\end{figure}

\section{Resumo e Conclusão do Capítulo}

Neste capítulo foram apresentados resultados visuais proporcionados pelo uso das ferramentas desenvolvidas durante o projeto, demonstrando a flexibilidade e usabilidade do sistema desenvolvido.
Também foram validados os resultados obtidos pelas avaliações objetivas e discutidos aspectos relevantes sobre as possíveis conclusões retiradas a partir dos valores obtidos.


\section{Conclusão}

% ------ estrutura
\subsection{Conclusões}
    \begin{frame}\frametitle{Completude dos Objetivos}
	Quanto às ferramentas:
	\begin{itemize}
		\item Geradores de artefatos:
		\begin{itemize}
			\item Configuráveis de forma flexível.
			\item Oferecem mais opções.
			\item Trabalham sobre arquivos YUV.
		\end{itemize}
		\item{Arquivo de degradação}
		\begin{itemize}
			\item Permite reprodução e controle sobre os artefatos.
			\item Economia de espaço.
		\end{itemize}
		\item Simulador:
		\begin{itemize}
			\item Dispensa \emph{interfaces} físicas.
			\item Configurações flexíveis.
		\end{itemize}
	\end{itemize}
    \end{frame}

	\begin{frame}\frametitle{Completude dos Objetivos}
	Quanto à \emph{Interface}:
	\begin{itemize}
		\item Funcionalidades semelhantes ao SASQV.
		\item Adaptações para agregar ferramentas e novas utilidades.
	\end{itemize}
	Quanto ao sistema:
	\begin{itemize}
		\item Portável, disponível em um arquivo \emph{.jar}.
		\item Ferramentas independentes, disponíveis em C++.
		\item Documentos tutoriais de ajuda. % diferencial do SASQV original, que nao tinha nada
	\end{itemize}
	\end{frame}
    
\subsection{Trabalhos Futuros}
    \begin{frame}\frametitle{Sugestões}
	\begin{itemize}
		\item Extender a funcionalidade do \emph{hardware}.
		\item Novas ferramentas independentes podem ser criadas.
		\item Adaptações no BD, agregando mais resultados.
		\item Aprimoramento da simulação.
	\end{itemize}
    \end{frame}
    
\subsection{Fim}
    \begin{frame}\frametitle{Obrigado!}
        \begin{table}[!h]
	        \begin{tabular}{lll}
                Brunno Braga & 41-8898-4111 & brunnobga@gmail.com \\
                Caio Andreatta & 41-8448-7157 & caioccaa@gmail.com \\
	        \end{tabular}
        \end{table}
    \end{frame}



%---------- Referencias ----------
\bibliography{reflatex} % geracao automatica das referencias a partir do arquivo reflatex.bib


%---------- Apendices (opcionais) ----------
%\apendice
%\input{guia.tex}
%\input{exemplodeuso.tex}


% ---------- Anexos (opcionais) ----------
%\anexo
%\chapter{Nome do Anexo}

%Use o comando {\ttfamily \textbackslash anexo} e depois comandos {\ttfamily \textbackslash chapter\{\}}
%para gerar t\'itulos de anexos.


% --------- Lista de siglas --------
%\textbf{* Observa\c{c}\~oes:} a lista de siglas nao realiza a ordenacao das siglas em ordem alfabetica
% Em breve isso sera implementado, enquanto isso:
%\textbf{Sugest\~ao:} crie outro arquivo .tex para siglas e utilize o comando \sigla{sigla}{descri\c{c}\~ao}.
%Para incluir este arquivo no final do arquivo, utilize o comando \input{arquivo.tex}.
%Assim, Todas as siglas serao geradas na ultima pagina. Entao, devera excluir a ultima pagina da versao final do arquivo
% PDF do seu documento.


%-------- Citacoes ---------
% - Utilize o comando \citeonline{...} para citacoes com o seguinte formato: Autor et al. (2011).
% Este tipo de formato eh utilizado no comeco do paragrafo. P.ex.: \citeonline{autor2011}

% - Utilize o comando \cite{...} para citacoeses no meio ou final do paragrafo. P.ex.: \cite{autor2011}



%-------- Titulos com nomes cientificos (titulo, capitulos e secoes) ----------
% Regra para escrita de nomes cientificos:
% Os nomes devem ser escritos em italico, 
%a primeira letra do primeiro nome deve ser em maiusculo e o restante em minusculo (inclusive a primeira letra do segundo nome).
% VEJA os exemplos abaixo.
% 
% 1) voce nao quer que a secao fique com uppercase (caixa alta) automaticamente:
%\section[nouppercase]{\MakeUppercase{Estudo dos efeitos da radiacao ultravioleta C e TFD em celulas de} {\textit{Saccharomyces boulardii}}
%
% 2) por padrao os cases (maiusculas/minuscula) sao ajustados automaticamente, voce nao precisa usar makeuppercase e afins.
% \section{Introducao} % a introducao sera posta no texto como INTRODUCAO, automaticamente, como a norma indica.


\end{document}
