%---------- Quarto Capítulo: Desenvolvimento ----------

\chapter{Desenvolvimento} %10 --- 20 pags

\section{Ferramentas Utilizadas}
\subsection{Sistema Operacional}

O desenvolvimento deste projeto foi totalmente realizado em ambiente Linux, em diferentes distribuições derivadas do Debian: Ubuntu 12.04 e Xubuntu 11.10.

\subsection{Ambientes de Desenvolvimento Integrado}

A primeira etapa de desenvolvimento planejada foi a de construir um modelo para a nova interface gráfica. Tendo em vista a linguagem Java como requisito, optou-se pelo \sigla{IDE}{Integrated Development Environment} NetBeans (versão 7.1.1) que possui uma ferramenta nativa específica para a construção de interfaces gráficas para usuário, a \sigla{GUI}{Graphical User Interface} Builder (Graphical User Interface). Esta ferramenta permite a construção de formulários no estilo drag-and-drop de containers existentes no Java, como por exemplo JFrame, JPanel, JButton, etc.

Inicialmente, o IDE Eclipse também foi utilizado com a finalidade de se realizar modificações nos projetos originais model e service. Posteriormente, para unificar o projeto em apenas um ambiente de desenvolvimento, os projetos originais foram importados no NetBeans.

\subsection{Outras ferramentas}

 g++ / vim / make

\subsection{Controle de Versões}

Em função de a equipe contar com dois integrantes desenvolvedores, existe a necessidade de se utilizar uma ferramenta para o controle de versões do projeto. Primeiramente, para realizar testes de algumas funcionalidades foi preciso modificar radicalmente o código o que, em certos casos, não gerou resultados satisfatórios, sendo necessário retornar à um ponto estável e funcional. Além deste motivo, o sistema de controle de versões escolhido, o Git, possui ferramentas de resolução de conflitos eficientes que auxiliam o desenvolvimento simultâneo.

O Git é um software livre e gratuito distribuído sob a licença \sigla{GNU}{General Public License} General Public License versão 2 \cite{git}.

Aliado a esta ferramenta, foi utilizado um serviço web de hospedagem de projetos que é organizado pelo sistema de controle de versões Git, o GitHub cujo endereço é http://github.com. Existem funcionalidades no estilo rede social como feeds, seguidores e gráficos diversos, bem como funcionalidades de projeto como visualização de pastas e códigos, gráficos de desempenho por usuário, por equipe, por período de desenvolvimento, frequência de código, histórico de modificações, entre outros \cite{githubabout}.

Na versão gratuita, que foi a utilizada, há uma exigência: que o código seja aberto \cite{githubabout}. Na versão paga, existem planos que permitem a criação de repositórios privados com times de desenvolvimento. Para ambas as versões o número de colaboradores é ilimitado, assim como o número de repositórios públicos.

\section{Ferramentas Desenvolvidas}

As ferramentas foram planejadas para uma utilização modular, ou seja, para serem utilizadas através da \emph{interface} do SASQV assim como através da linha de comando, de forma independente. Um dos requisitos destas ferramentas é a de que sejam aplicadas sobre vídeos no formato YUV com subamostragem 4:2:0 \emph{progressive}.

\subsection{Raffle}

A ferramenta raffle tem como objetivo gerar elementos aleatórios multi-dimensionais no domínio dos naturais, armazenado-os em arquivo.

No arquivo gerado, cada coluna representa uma dimensão que deve seguir alguma distribuição de probabilidade, informada via parâmetros. Dentre as distribuições possíveis e seus respectivos parâmetros tem-se: 

\begin{itemize}
	\item Distribuição constante: a dimensão correspondente terá valor constante c para todos os elementos gerados.
	\item Distribuição uniforme: a dimensão correspondente terá valores uniformemente gerados num intervalo [a, b).
	\item Distribuição normal: a dimensão correspondente terá valores gerados conforme uma distribuição normal, de média m e desvio padrão d.
	\item Distribuição triangular: a dimensão correspondente terá valores gerados conforme uma distribuição triangular no intervalo [a, b) com pico em c.
\end{itemize}

A Tabela \ref{tab:distparam} sintetiza os parâmetros que devem ser informados conforme o tipo de distribuição desejada.

\begin{table}[!h]
	\centering
	\caption{Distribuições e respectivos parâmetros para a execução da ferramenta raffle.}
	\label{tab:distparam}
	\begin{tabular}{l|l|l|l|l}
		\hline
		Distribuição & \multicolumn{4}{c}{Parâmetros} \\
		\hline
		Constante  & -d ou -u & constant   & -p ou -r & a \\
		Uniforme   & -d ou -u & uniform	   & -p ou -r & a,b \\
		Normal     & -d ou -u & normal	   & -p ou -r & m,d \\
		Triangular & -d ou -u & triangular & -p ou -r & a,b,c \\
		\hline
	\end{tabular}
\end{table}

Adicionalmente aos parâmetros de cada distribuição, a primeira coluna possui uma característica temporal que pode ser ativada, permitindo a repetição de determinado elemento ao longo de um intervalo constante ou aleatório. Para este sorteio é possível utilizar qualquer distribuição descrita acima.

O comando para utilizar a ferramenta e seus parâmetros são descritos a seguir:

\begin{table}[!h]
	\begin{tabular}{llll}
	./raffle & & \\ 
	& \texttt{--output} & \texttt{-o}  & arquivo\_de\_saída \\
	& \texttt{--durationdist} & \texttt{-u}  & tipo de distribuição da duração \\
	& \texttt{--durationparams} & \texttt{-r}  & parâmetros da distribuição da duração \\
	& \texttt{--elements} & \texttt{-e}  & elementos a serem gerados \\
	& \texttt{--dist} & \texttt{-d}  & tipo de distribuição da primeira dimensão \\
	& \texttt{--params} & \texttt{-p}  & parâmetros da distribuição da primeira dimensão \\
	& ... & \\
	& \texttt{--dist} & \texttt{-d}  & tipo de distribuição da n-ésima dimensão \\
	& \texttt{--params} & \texttt{-p}  & parâmetros da distribuição da n-ésima dimensão \\
	& \texttt{--help} & \texttt{-h}  & menu de ajuda \\
	\end{tabular}
\end{table}

Observações:
\begin{itemize}
    \item[-] Todo parâmetro -d deve ser imediatamente sucedido por um parâmetro -p
    \item[-] O parâmetro -u deve ser imediatamente sucedido por um parâmetro -r
    \item[-] Cada conjunto (-d -p) representa uma dimensão, ou uma coluna, a ser gerada.
\end{itemize}

Exemplos de utilização através da linha de comando são descritos abaixo. A diferença para a \emph{interface} gráfica reside apenas no fornecimento dos dados, uma vez que a \emph{interface} deve executar o mesmo comando para gerar o arquivo.

No comando a seguir, será gerado um arquivo com nome raffleout.rff contendo 30 elementos. Para este exemplo, será utilizado um valor de duração constante e igual a 1, no próximo exemplo este conceito de duração será melhor detalhado. A primeira dimensão deve seguir uma distribuição uniforme dentro do intervalo [1, 5), a segunda dimensão deve ter valor constante 3 para todos os elementos gerados e a terceira e última dimensão deve seguir uma distribuição triangular no intervalo [3, 20) com pico no ponto 10.

./raffle -o raffleout.rff -u constant -r 1 -e 30 -d uniform -p 1,5 -d constant -p 3 -d triangular -p 3,20,10

O resultado de uma execução deste comando pode ser observado na Tabela \ref{tab:raffleresult1}, a seguir:

\begin{table}[!h]
	\centering
	\caption{Resultado de execução do commando raffle para exemplo 1.}
	\label{tab:raffleresult1}
	\begin{tabular}{|l|l|l|l|l|l|}
		\hline
		4 3 6 & 1 3 16 & 1 3 16 & 1 3 18 & 3 3 16 & 3 3 11 \\
		3 3 18 & 3 3 11 & 2 3 6 & 2 3 13 & 2 3 10 & 3 3 14 \\
		1 3 11 & 2 3 11 & 4 3 9 & 4 3 11 & 2 3 15 & 3 3 10 \\
		1 3 9 & 3 3 11 & 2 3 7 & 1 3 7 & 4 3 5 & 1 3 4 \\
		1 3 10 & 3 3 11 & 4 3 10 & 4 3 11 & 2 3 12 & 3 3 6 \\
		\hline
	\end{tabular}
\end{table}

No comando abaixo, será gerado um arquivo de nome raffleout2.rff com 50 elementos de duas dimensões: a primeira seguirá uma distribuição uniforme dentro do intervalo [1,5) e a segunda seguirá uma distribuição normal com média 2 e desvio padrão 1:

./raffle -o raffleout2.rff -u constant -r 3 -e 50 -d uniform -p 1,5 -d normal -p 2,1

A duração, neste exemplo, tem duração constante 3, ou seja, para todo elemento gerado, ele deverá ter duração de 3 elementos no total. O efeito da duração pode ser observado na figura, onde o primeiro elemento (4 1) tem duração 3 tendo a primeira dimensão como referência temporal: (4 1), (5 1) e (6 1). A geração de elementos prossegue até que o número de elementos gerados atinja o desejado, informado no parâmetro -e. Um resultado importante, que pode ser observado na Tabela \ref{tab:raffleresult2}, é que a duração não ultrapassa os limites definidos para a primeira coluna fazendo com que nem todos os artefatos tenham a duração estabelecida.

\begin{table}[!h]
	\centering
	\caption{Resultado de execução do commando raffle para exemplo 2.}
	\label{tab:raffleresult2}
	\begin{tabular}{|l|l|l|l|l|l|l|l|l|l|}
		\hline
		3 2 & 4 2 & 3 1 & 4 1 & 4 1 & 2 1 & 3 1 & 4 1 & 3 2 & 4 2 \\
		3 4 & 4 4 & 2 3 & 3 3 & 4 3 & 4 1 & 1 2 & 2 2 & 3 2 & 4 1 \\
		4 0 & 3 2 & 4 2 & 3 1 & 4 1 & 4 1 & 2 3 & 3 3 & 4 3 & 1 2 \\
		2 2 & 3 2 & 1 0 & 2 0 & 3 0 & 1 2 & 2 2 & 3 2 & 2 2 & 3 2 \\
		4 2 & 2 3 & 3 3 & 4 3 & 1 0 & 2 0 & 3 0 & 1 2 & 2 2 & 3 2 \\
		\hline
	\end{tabular}
\end{table}

%TODO
Na aplicação do efeito de blocagem, por exemplo, a posição de cada pixel no vídeo é representada por uma tupla de três dimensões: o frame F em que o pixel aparece, a posição X em relação ao eixo vertical e a posição Y em relação ao eixo horizontal. Na simulação de rede, os vídeos são representados por uma sequência de pacotes,

A ferramenta raffle foi inicialmente utilizada de forma interna nas ferramentas block e netsim. %TODO

\subsection{Block}

Esta ferramenta é responsável por aplicar o efeito de blocagem sobre o vídeo desejado. Os parâmetros que devem ser fornecidos são descritos a seguir:

\begin{table}[!h]
	\begin{tabular}{llll}
	./block & & \\ 
	& \texttt{--input} & \texttt{-i}  & arquivo\_de\_entrada \\
	& \texttt{--output} & \texttt{-o}  & arquivo\_de\_saída \\
	& \texttt{--size} & \texttt{-s}  & dimensões do vídeo de entrada no formato WxH (largura x altura em pixels) \\
	& \texttt{--window} & \texttt{-w}  & tamanho da janela da \sigla{DCT}{Discrete Cosine Transform} \\
	& \texttt{--levelsdct} & \texttt{-l}  & níveis DCT à serem eliminados, separados por ponto-e-vírgula \\
	& \texttt{--rafflelist} & \texttt{-r}  & arquivo\_raffle \\
	& \texttt{--help} & \texttt{-h}  & menu de ajuda \\
	\end{tabular}
\end{table}

A partir do vídeo desejado, deve-se escolher o nome do vídeo resultante e informar as dimensões do vídeo original, o tamanho da janela a ser utilizada na DCT, os subníveis de energia que deverão ser eliminados (zerados) e o nome do arquivo raffle a ser utilizado.

O arquivo raffle de entrada deve sempre conter três colunas. A primeira diz respeito ao \emph{frame} em que será aplicado o artefato. A segunda e a terceira coluna devem ter como limite o resultado da divisão da largura e altura do vídeo pelo tamanho da janela desejada, respectivamente, pois seus valores indicam qual grupo de pixels será afetado e nao o pixel inicial de aplicação da DCT.

Os subníveis de energia (diagonais da matriz resultante da DCT) que serão cancelados dependem diretamente do tamanho da janela a ser aplicada. Seja uma janela de dimensões AxA, existem N = 2A-1 níveis de energia. O maior nível, 2A-1, representa o nível de energia \sigla{DC}{Differential Coding} (\emph{Differential Coding}) (a média dos valores do bloco). Nos demais níveis, conforme o nível diminui, a energia diminui porém a frequência aumenta. São conhecidos como coeficientes \sigla{AC}{Arithmetic Coding} (\emph{Arithmetic Coding}). A Figura \ref{fig:niveisdct} apresenta a organização dos níveis de energia numa janela de dimensões 8x8 pixels.

\begin{figure}[!htb]
	\centering
	\includegraphics[width=0.5\textwidth]{./imgs/niveisdct.png}
	\caption{Diagonais de energia do bloco DCT.}
	\label{fig:niveisdct}
	\fonte{\cite{}}
\end{figure}

O arquivo gerado deve ter as mesmas características do original: tamanho em bytes, largura e altura.

Como exemplo de utilização temos que na execução do comando a seguir, o arquivo saida.yuv de dimensões 352x288 será gerado. O vídeo original será submetido aos valores contidos no arquivo raffle\_352x288.rff e cada janela de dimensões 8x8 terá os níveis DCT (ou diagonais) 1,2,4,6,7,15 zerados.

./block -i entrada.yuv -o saida.yuv -s 352x288 -w 8 -l 1,2,4,6,7,15 -r raffle\_352x288.rff

A única restrição se deve aos valores do arquivo raffle, como já mencionado. Este deve conter três colunas, em que a primeira, segunda e terceira colunas devem ter como limites superiores, respectivamente, o número de \emph{frames} do vídeo original e as dimensões largura e altura divididas pelo tamanho da janela (352/8 = 44 e 288/8 = 36).

\subsection{blur}

A ferramenta blur aplica o artefato de borramento utilizando para tanto um filtro da média ou um filtro da mediana, de dimensão configurável. O arquivo de saída é gerado aplicando-se o artefato configurado em todos os frames do vídeo original. Os parâmetros de configuração e respectivas descrições podem ser observados adiante:

\begin{table}[!h]
	\begin{tabular}{llll}
	./blur & & \\ 
	& \texttt{--input} & \texttt{-i}  & arquivo\_de\_entrada \\
	& \texttt{--output} & \texttt{-o}  & arquivo\_de\_saída \\
	& \texttt{--size} & \texttt{-s}  & dimensões do vídeo de entrada no formato WxH (largura por altura em pixels) \\
	& \texttt{--blur} & \texttt{-b}  & tipo do filtro a ser aplicado \\
	& \texttt{--window} & \texttt{-w}  & tamanho do filtro \\
	& \texttt{--help} & \texttt{-h}  & menu de ajuda \\
	\end{tabular}
\end{table}

Numa possível execução da ferramenta como indicada abaixo, o filtro da média de dimensões 5x5 pixels será aplicado no vídeo original, gerando o vídeo degradado saida.yuv de dimensões iguais às do original (\sigla{Full-HD}{Resolução Full-HD de 1080 linhas contendo 1980 pixels cada.}).

./blur -i entrada.yuv -o saida.yuv -s 1920x1080 -b average -w 5

Observações:
\begin{itemize}
    \item[-] para utilizar o filtro da média, deve-se fornecer o parâmetro -b \emph{average}, para o filtro da mediana deve-se fornecer o parâmetro -b \emph{median};
    \item[-] a menor dimensão para qualquer tipo de filtro é 3;
\end{itemize}

\subsection{NetSim}

A ferramenta Netsim foi desenvolvida buscando simular degradações ocasionadas pelo processo de decodificação de um streaming de vídeo onde houve perda de informação nas camadas de transporte, rede ou enlace. O Netsim desconsidera, portanto, os artefados decorrentes do processo de codificação e encapsulamento do vídeo.

As simulações efetuadas pela ferramenta atuam sobre vídeos encapsulados no formato MPEG Transport Stream, conforme definido em \cite{} %TODO referencia para ITU-T Rec. H.222.0
e consistem no descarte controlado de porções de informação. 
A entidade de dados a ser descartada pode ser tanto um único TS quanto o equivalente a um pacote UDP da camada de transporte, o qual comporta usualmente sete unidades TS.
O descarte é controlado precisamente por meio de um arquivo de configuração fornecido como paramêtro ao programa, podendo este ser gerado pela ferramenta Raffle descrita em \ref{}. %TODO referencia pra raffle.
Este arquivo de configuração deve conter duas colunas com valores numéricos inteiros. 
Cada linha será processada sequencialmente, sendo que o primeiro número indica quantas entidades serão transportadas com sucesso e o segundo quantas serão descartadas. 
A Figura \ref{fig:netsim} ilustra a sequência.

\begin{figure}[!htb]
	\centering
	\includegraphics[width=0.9\textwidth]{./imgs/netsim.png}
	\caption{Ilustração do arquivo de configuração de descartes da ferramenta Netsim.}
	\label{fig:netsim}
	\fonte{Autoria Própria.}
\end{figure}

Neste caso, \emph{A} e \emph{C} indicam o número de entidades que atingem seu destino com sucesso e \emph{B} e \emph{D} indicam quantas serão descartadas. 
Sendo assim o streaming de vídeo da Figura \ref{fig:netsim}  teria seus primeiros \emph{A} pacotes intactos, seguidos de \emph{B} perdidos, mais \emph{C} pacotes intactos e ainda \emph{D} perdidos novamente.

\begin{description}
	\item[\texttt{--input}] \hfill \\
		Define o caminho (absoluto ou relativo) e arquivo do vídeo a ser processado pela simulação.
	\item[\texttt{--output}] \hfill \\
		Define o caminho(absoluto ou relativo) e arquivo onde será armazenado o vídeo resultante.
	\item[\texttt{--ts}] \hfill \\
		(Opcional) Se presente, a entidade de descarte considerada é um TS, caso contrário a unidade é um pacote UDP.
	\item[\texttt{--raffle}] \hfill \\
		Indica o caminho (absoluto ou relativo) e arquivo a ser usado como configuração de descartes.
\end{description}


\subsection{Metric}

A ferramenta Metric se trata da implementação de três métricas objetivas, as mesma encontradas na implementação do SASQV:

\begin{itemize}
	\item MSE
	\item PSNR
	\item MSSIM
\end{itemize}

Implementada em C++, também na forma de uma ferramenta \emph{stand-alone}, seu funcionamento é baseado em verificar disparidades entre dois vídeos fornecidos seguindo uma das métricas implementadas e fornecer um resultado numérico. 
Os vídeos a serem comparados devem possuir as mesmas dimensões e número de frames para serem passíveis de comparação. 
A ferramenta Metric pode receber os seguintes parametros:

\begin{description}
	\item[\texttt{--input}] \hfill \\
		Define o caminho(absoluto ou relativo) e arquivo onde um dos vídeos a serem comparados se encontra.
	\item[\texttt{--reference}] \hfill \\
		Define o caminho(absoluto ou relativo) e arquivo onde o segundo vídeo a ser comparado se encontra.
	\item[\texttt{--size}] \hfill \\
		Define as dimensões (em pixels) dos vídeos a serem comparados. Deve ser fornecida no formato 'largura'x'altura'.
	\item[\texttt{--metric}] \hfill \\
		Define qual métrica será utilizada na comparação entre vídeos, sendo uma string entre MSE, PSNR ou MSSIM.
	\item[\texttt{--window}] \hfill \\
		Define qual o tamanho da janela (em pixels) a ser usado se a métrica adotada for MSSIM.
\end{description}

\section{Interface Gráfica}
\subsection{Sessão}
\subsection{Ferramentas}
\subsection{Resultados}
\subsection{Configurações}
\subsection{Ajuda}

\section{Diagramas de Classes}

\section{Diagramas de Sequências}
\section{Considerações}
