\chapter{Recursos de \emph{Hardware} e \emph{Software}}

\section{Recursos de \emph{Hardware}}

Este projeto não visa desenvolver ou aprimorar equipamentos físicos existentes no SASQV, portanto os recursos de \emph{hardware} permanecem inalterados em relação ao projeto de \cite{sasqv}. Como principais recursos podemos destacar o conjunto de dispositivos de entrada provido de comunicação sem fio, um dispositivo receptor capaz de se comunicar com os dispositivos de entrada e também um computador possuindo atributos de processamento e armazenamento capazes de receber um sistema gerenciador de banco de dados e suprir os requisitos necessários para reprodução de vídeo digital em qualidade \sigla{HD}{High Definition}.

\section{Recursos de \emph{Software}}

Sendo este continuidade do sistema SASQV, as tecnologias \cite{java} e \cite{mysql} serão utilizadas no projeto, assim como o \sigla{IDE}{Integrated Development Environment} \cite{eclipse}. Novidades serão introduzidas a partir da utilização da linguagem de programação C++ para construção de ferramentas auxiliares. O projeto também terá versionamento de código e documentos disponibilizados em \cite{githubcode, githubdoc} por meio da ferramenta \cite{git}.

\section{Viabilidade}
