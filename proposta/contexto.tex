\chapter{Contexto}

Este projeto dá continuidade ao projeto desenvolvido por \cite{sasqv}, apresentado como trabalho de conclusão de curso dos programas de graduação em Engenharia Eletrônica e Engenharia de Computação da Universidade Tecnológica Federal do Paraná, e visa aprimorar a ferramenta SASQV para atender as metodologias utilizadas em ambiente acadêmico e no mercado, produzindo artefatos que se assimilem àqueles encontrados cotidianamente e descritos em \cite{albini, vqeg}. 
Sendo assim, o objetivo desta continuação é preencher as funcionalidades deixadas inacabadas no projeto anterior, criando uma ferramenta mais completa que possa cobrir os procedimentos necessários para uma sessão de avaliação subjetiva de vídeo.
